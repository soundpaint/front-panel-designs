\NeedsTeXFormat{LaTeX2e}
\documentclass[tikz,margin=0mm,12pt,utf8,a4paper]{standalone}

\usepackage{marvosym}
\usepackage{lmodern}
\usepackage{setspace}
\usepackage[utf8]{inputenc}
\usepackage[T1]{fontenc}
\usepackage{soul}
\usetikzlibrary{positioning}
\usepgflibrary{shadings}

\begin{document}

\pagestyle{empty}

% parameters: xshift, yshift
\newcommand{\drawpolarity}[2]{
  \begin{scope}[xshift=#1,yshift=#2,line width=0.2mm]
    \draw[fill=black] (0,0) circle (.4mm);
    \begin{scope}[xshift=1.0mm*cos(-150),yshift=1.0mm*sin(-150),line cap=round]
      \draw (0,0) arc (-150:+150:1.0mm);
    \end{scope}
    \draw[line cap=round] (-1.5mm,0) -- (0,0);
    \draw (-3.0mm,0) -- (-2.0mm,0);
    \draw (-2.5mm,-0.5mm) -- (-2.5mm,+0.5mm);
  \end{scope}
}

% parameters: xshift, yshift, top label, bottom line #1, bottom line #2
\newcommand{\drawdcsocket}[5]{
  \begin{scope}[xshift=#1,yshift=#2]
    \draw[fill=white] (0,0) circle (4mm);
    \draw[fill=black] (0,0) circle (.25mm);
    \node
    at (0,+0.7cm) {\fontsize{10}{0}\textbf{\sf #3}};
    \node (line1)
    at (0,-0.7cm) {\fontsize{8}{0}\textbf{\sf #4}};
    \node[below=3mm of line1.west,anchor=west]
    {\fontsize{8}{0}\textbf{\sf #5}};
    \drawpolarity{-6mm}{-3.5mm}
  \end{scope}
}

\newcommand{\dccurrent}{
  % workaround for missing Unicode symbol “⎓” (U+2393)
  \raisebox{1mm}{\Beam}
}

\newcommand{\accurrent}{
  % workaround for missing Unicode symbol“⏦” (U+23E6)
  \normalsize\st{$\sim$}
}

% parameters: xshift, yshift
\newcommand{\drawbackpanel}[2]{
  \begin{scope}[xshift=#1,yshift=#2]
    \drawdcsocket{-1.6cm}{0}{DC/AC Input}{4V$\ldots$24V\accurrent}{5V$\ldots$30V\dccurrent}
    \drawdcsocket{+1.6cm}{0}{DC Output}{1.5V$\ldots$24V\dccurrent}{max. 2000mA}
    \draw (-2.90cm,-1.25cm) rectangle (+2.90cm,+1.05cm);
  \end{scope}
}

% parameters: xshift, yshift, label
\newcommand{\drawgauge}[3]{
  \begin{scope}[xshift=#1+1.25cm, yshift=#2,
      text height=0.0cm, text depth=0.0cm, anchor=base]
    \def\radius{1cm}
    \def\colorwheelwidth{0.8mm}
    \def\innerradius{\radius-0.5*\colorwheelwidth}
    \def\minorticklen{0.8mm}
    \def\majorticklen{1.2mm}
    \def\maxticklen{1.6mm}
    \def\labelradius{1.3cm}
    \def\sqrthalf{0.707}

    % add subtle gray tone as background
    \shade[shading=radial, inner color=white, outer color=gray!15]
    (0,0) circle (\radius);

    % drill-hole
    \draw[fill=black] (0,0) circle (.1mm);
    \draw (0,0) circle (3.5mm);

    % partial dial gauge outline
    \draw[black] (\sqrthalf*\radius,-\sqrthalf*\radius)
    arc[start angle=-45, end angle=225, radius=\radius];
    \node[draw, circle, inner sep=.1mm,
      label={[label distance=1.4cm,
          font=\sf\bf\fontsize{10}{0}]90:\textbf{\sf#3}}]
    (anchor) at (0,0) {};

    % color wheel
    \iftrue
    \draw[shading=color wheel]
    (225:{\innerradius-.5*\colorwheelwidth})
    arc [start angle=225, delta angle=-270,
      radius={\innerradius-.5*\colorwheelwidth}] --++
    ({225-270}:\colorwheelwidth)
    arc [start angle={225-270}, delta angle=270,
      radius={\innerradius+.5*\colorwheelwidth}] --
    cycle;
    \fi

    % short thin line at every 3 degrees
    \foreach \x in {-45,-42,...,225}
    \draw[line width=0.1mm] (\x:\radius-\minorticklen) -- (\x:\radius);

    % long thin line at every 12 degrees
    \foreach \x in {-45,-33,...,219}
    \draw[line width=0.1mm] (\x:\radius-\majorticklen) -- (\x:\radius);

    % very long fat line at every 24 degrees
    \foreach \x in {-45,-21,...,219}
    \draw[line width=0.2mm] (\x:\radius-\maxticklen) -- (\x:\radius);

    % labels
    \foreach \x in {1.5,4,6,...,24}{
      \def\volt{\number\numexpr\x*0.5\relax}
      \node[rotate=0, font=\fontsize{8}{0},xshift=+0.5mm,yshift=-1.5mm]
      at (243-\x*12:\labelradius) {\textbf{\sf \x{}V}};
    }
  \end{scope}
}

% parameters: xshift, yshift, label
\newcommand{\drawswitch}[3]{
  \begin{scope}[xshift=#1+0.25cm,yshift=#2]
    % drill-hole
    \draw[fill=black] (0,0) circle (.1mm);
    \draw (0,0) circle (2.5mm);

    % labels
    \node
    at (0,1.285cm) {\fontsize{10}{0}\textbf{\sf #3}};
    \node
    at (0,+0.7cm) {\fontsize{8}{0}\textbf{\sf Off}};
    \node
    at (0,-0.7cm) {\fontsize{8}{0}\textbf{\sf On}};
  \end{scope}
}

% parameters: xshift, yshift, label
\newcommand{\drawled}[3]{
  \begin{scope}[xshift=#1,yshift=#2]
    % drill-hole
    \draw[fill=black] (0,0) circle (.1mm);
    \draw (0,0) circle (4.0mm);

    % label
    \node
    at (0,-0.7cm) {\fontsize{8}{0}\textbf{\sf #3}};
  \end{scope}
}

% parameters: xshift, yshift
\newcommand{\drawfrontpanel}[2]{
  \begin{scope}[xshift=#1,yshift=#2]
    \drawswitch{-2.5cm}{0}{}
    \drawgauge{-1.5cm}{-0.1cm}{Output Voltage}
    \drawled{+1.9cm}{+1.3cm}{Power}
    \draw (-2.7cm,-1.30cm) rectangle (+2.5cm,+1.80cm);
  \end{scope}
}

\begin{tikzpicture}[scale=1.0,line width=0.1mm]
  \foreach \y in {0,1,2,3}{
    \drawfrontpanel{0}{\y*4.5cm}
    \drawbackpanel{7cm}{\y*4.5cm}
  }
\end{tikzpicture}

\end{document}
