\NeedsTeXFormat{LaTeX2e}
\documentclass[tikz,margin=0mm,12pt,utf8,a4paper]{standalone}

\usepackage{lmodern}
\usepackage[utf8]{inputenc}
\usepackage[T1]{fontenc}
\usepgflibrary{shadings}
%\usepackage[showframe]{geometry}

\begin{document}

\pagestyle{empty}

\def\radius{1cm}
\def\colorwheelwidth{0.8mm}
\def\innerradius{\radius-0.5*\colorwheelwidth}
\def\minorticklen{0.8mm}
\def\majorticklen{1.2mm}
\def\maxticklen{1.6mm}
\def\labelradius{1.3cm}
\def\sqrthalf{0.707}

% parameters: xshift, yshift, label
\newcommand{\drawgauge}[3]{
  \begin{scope}[xshift=#1+1.25cm, yshift=#2,
      text height=0.0cm, text depth=0.0cm, anchor=base]
    % add subtle gray tone as background
    \shade[shading=radial, inner color=white, outer color=gray!15]
    (0,0) circle (\radius);

    % drill-hole
    \draw[fill=black] (0,0) circle (.1mm);
    \draw (0,0) circle (3mm);

    % partial dial gauge outline
    \draw[black] (\sqrthalf*\radius,-\sqrthalf*\radius)
    arc[start angle=-45, end angle=225, radius=\radius];
    \node[draw, circle, inner sep=.1mm,
      label={[label distance=1.0cm,
          font=\sf\bf\fontsize{10}{0}]90:\textbf{\sf#3}}]
    (anchor) at (0,0) {};

    % color wheel
    \iftrue
    \draw[shading=color wheel]
    (225:{\innerradius-.5*\colorwheelwidth})
    arc [start angle=225, delta angle=-270,
      radius={\innerradius-.5*\colorwheelwidth}] --++
    ({225-270}:\colorwheelwidth)
    arc [start angle={225-270}, delta angle=270,
      radius={\innerradius+.5*\colorwheelwidth}] --
    cycle;
    \fi

    % short thin line at every 3 degrees
    \foreach \x in {-45,-42,...,225}
    \draw[line width=0.1mm] (\x:\radius-\minorticklen) -- (\x:\radius);

    % long thin line at every 15 degrees
    \foreach \x in {-45,-30,...,225}
    \draw[line width=0.1mm] (\x:\radius-\majorticklen) -- (\x:\radius);

    % very long fat line at every 30 degrees
    \foreach \x in {-45,-15,...,225}
    \draw[line width=0.2mm] (\x:\radius-\maxticklen) -- (\x:\radius);

    % labels
    \node[rotate=0, font=\fontsize{8}{0}]
    at (225:\labelradius) {\textbf{\sf Min}};
    \node[rotate=0, font=\fontsize{8}{0}]
    at (-45:\labelradius) {\textbf{\sf Max}};
  \end{scope}
}

% parameters: xshift, yshift, label
\newcommand{\drawswitch}[3]{
  \begin{scope}[xshift=#1+0.25cm,yshift=#2]
    % drill-hole
    \draw[fill=black] (0,0) circle (.1mm);
    \draw (0,0) circle (2mm);

    % labels
    \node
    at (0,1.285cm) {\fontsize{10}{0}\textbf{\sf #3}};
    \node
    at (0,+0.7cm) {\fontsize{8}{0}\textbf{\sf Off}};
    \node
    at (0,-0.7cm) {\fontsize{8}{0}\textbf{\sf On}};
  \end{scope}
}

% parameters: xshift, width, label
\newcommand{\drawbracket}[3]{
  \begin{scope}[xshift=#1, yshift=1.8cm, line width=0.5mm, line cap=round]
    \draw (0,0) -- (#2,0)
    node [midway, fill=white, align=center]
    {\fontsize{10}{0}\textbf{\sf{ #3 }}};
    \draw (0,0.0cm) -- (0,-0.2cm);
    \draw (#2,0.0cm) -- (#2,-0.2cm);
  \end{scope}
}

% parameters: xshift, yshift
\newcommand{\drawpowerarea}[2]{
  \begin{scope}[xshift=#1,yshift=#2]
    \drawbracket{0}{3.30cm}{Power}
    \drawgauge{0}{0}{Volume}
    \drawswitch{2.75cm}{-0.12cm}{}

    % LED
    \draw[fill=black] (2.95cm,1.28cm) circle (.1mm);
    \draw (2.95cm,1.28cm) circle (3.0mm);
  \end{scope}
}

% parameters: xshift, yshift
\newcommand{\drawnfmodulationarea}[2]{
  \begin{scope}[xshift=#1,yshift=#2]
    \drawbracket{0}{5cm}{NF Modulation}
    \drawgauge{0}{0}{Depth}
    \drawgauge{2.5cm}{0}{Hacker Pitch}
  \end{scope}
}

% parameters: xshift, yshift
\newcommand{\drawhfmodulationarea}[2]{
  \begin{scope}[xshift=#1,yshift=#2]
    \drawbracket{0}{5.80cm}{HF Modulation}
    \drawgauge{0}{0}{Depth}
    \drawgauge{2.5cm}{0}{Hacker Pitch}
    \drawswitch{5.25cm}{-0.12cm}{}
  \end{scope}
}

% parameters: xshift, yshift
\newcommand{\drawpanel}[2]{
  \begin{scope}[xshift=#1,yshift=#2]
    \drawpowerarea{5.5cm}{0}
    \drawnfmodulationarea{0}{0}
    \drawhfmodulationarea{0}{-3.2cm}
  \end{scope}
}

\begin{tikzpicture}[scale=1.0,line width=0.1mm]
  \foreach \x in {0}
  \foreach \y in {0,1,2}{
    \drawpanel{\x*9cm}{\y*7.2cm}
  }
\end{tikzpicture}

\end{document}
