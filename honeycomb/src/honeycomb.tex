\NeedsTeXFormat{LaTeX2e}
\documentclass[tikz,margin=0mm,12pt,a4paper]{standalone}

\begin{document}
\pagestyle{empty}
\newcommand\dy{0.2887} % sqrt(3) / 6
\begin{tikzpicture}
  % Most simple implementation, but not aligned with paper layout:
  %
  %\foreach \y in {0,1,...,10}{
  %  \foreach \x in {0,1,...,10}{
  %    \draw (\x+0.5*\y,\y*\dy) -- (\x+0.3333+0.5*\y,\y*\dy);
  %    \draw (\x+0.3333+0.5*\y,\y*\dy) -- (\x+0.5*\y+0.5,\y*\dy+\dy);
  %    \draw (\x+0.5*\y,\y*\dy) -- (\x-0.6667+0.5*\y+0.5,\y*\dy+\dy);
  %  }
  %}
  \foreach \x in {0,1,...,20}{
    \message{.}
    \foreach \y in {0,2,...,100}{
      \draw (\x,       \y*\dy-\dy) -- (\x+0.3333,\y*\dy-\dy);
      \draw (\x+0.3333,\y*\dy-\dy) -- (\x+0.5000,\y*\dy);
      \draw (\x,       \y*\dy-\dy) -- (\x-0.1667,\y*\dy);
      \draw (\x+0.5000,\y*\dy)     -- (\x+0.8333,\y*\dy);
      \draw (\x-0.1667,\y*\dy)     -- (\x,       \y*\dy+\dy);
      \draw (\x+0.5000,\y*\dy)     -- (\x+0.3333,\y*\dy+\dy);
    }
    % Optionally: Draw circles to see how they intersect the
    % honeycombs (if you want to create something like Bresenham's
    % circle drawing algorithm, but for a honeycomb raster).
    %
    %\draw (10.1667,50*\dy) circle (10pt);
    %\draw (10.1667,50*\dy) circle (100pt);
    %\draw (10.1667,50*\dy) circle (200pt);
  }
\end{tikzpicture}
\end{document}

%  Local Variables:
%    coding:utf-8
%    mode:LaTeX
%  End:
